%% Thesis: Fine-tuned YOLO11 for Maritime Vessel Detection in the Bosphorus Strait
%% Author: Recep Ertuğrul Ekşi
%% Advisor: Rowanda D. Ahmed
%% Institution: Üsküdar University, Istanbul
%% Date: 2025

\documentclass[12pt,a4paper,oneside]{report}

%% ============================================================================
%% PACKAGES
%% ============================================================================

% Encoding and fonts
\usepackage[utf8]{inputenc}
\usepackage[T1]{fontenc}
\usepackage{times}  % Times New Roman font

% Page layout
\usepackage[left=3.5cm, right=2.5cm, top=2.5cm, bottom=2.5cm]{geometry}
\usepackage{setspace}
\onehalfspacing  % 1.5 line spacing

% Graphics and figures
\usepackage{graphicx}
\graphicspath{{../Paper - Computer Science Journal/figures/}{./figures/}}
\usepackage{float}
\usepackage{subcaption}

% Tables
\usepackage{booktabs}
\usepackage{multirow}
\usepackage{array}
\usepackage{longtable}
\usepackage{tabularx}

% Math
\usepackage{amsmath}
\usepackage{amssymb}
\usepackage{amsfonts}

% Colors and hyperlinks
\usepackage[table,xcdraw]{xcolor}
\usepackage[hidelinks]{hyperref}
\usepackage{url}

% Code listings
\usepackage{listings}
\lstset{
    basicstyle=\ttfamily\footnotesize,
    breaklines=true,
    frame=single,
    numbers=left,
    numberstyle=\tiny,
    captionpos=b
}

% Bibliography
\usepackage[numbers,sort&compress]{natbib}

% Appendix
\usepackage[toc,page]{appendix}

% Headers and footers
\usepackage{fancyhdr}
\pagestyle{fancy}
\fancyhf{}
\fancyhead[L]{\leftmark}
\fancyhead[R]{\thepage}
\renewcommand{\headrulewidth}{0.4pt}

% Chapter and section formatting
\usepackage{titlesec}
\titleformat{\chapter}[display]
    {\normalfont\huge\bfseries}
    {\chaptertitlename\ \thechapter}
    {20pt}
    {\Huge}
\titlespacing*{\chapter}{0pt}{-30pt}{40pt}

% Table of contents depth
\setcounter{tocdepth}{3}
\setcounter{secnumdepth}{3}

% Custom column types
\newcolumntype{L}[1]{>{\raggedright\arraybackslash}p{#1}}
\newcolumntype{C}[1]{>{\centering\arraybackslash}p{#1}}
\newcolumntype{R}[1]{>{\raggedleft\arraybackslash}p{#1}}

%% ============================================================================
%% DOCUMENT INFORMATION
%% ============================================================================

\newcommand{\thesistitle}{Fine-tuned YOLO11 for Maritime Vessel Detection in the Bosphorus Strait: A Comparative Study}
\newcommand{\authorname}{Recep Ertuğrul Ekşi}
\newcommand{\advisorname}{Dr. Rowanda D. Ahmed}
\newcommand{\department}{Department of Computer Engineering}
\newcommand{\university}{Üsküdar University}
\newcommand{\thesisdate}{2025}

%% ============================================================================
%% DOCUMENT BEGIN
%% ============================================================================

\begin{document}

%% ----------------------------------------------------------------------------
%% FRONT MATTER
%% ----------------------------------------------------------------------------
\frontmatter
\pagenumbering{roman}

% Title Page
\begin{titlepage}
    \centering
    \vspace*{1cm}
    
    {\Large \university \par}
    \vspace{0.5cm}
    {\large \department \par}
    
    \vspace{3cm}
    
    {\huge\bfseries \thesistitle \par}
    
    \vspace{3cm}
    
    {\Large Master's Thesis \par}
    
    \vspace{2cm}
    
    {\Large \authorname \par}
    
    \vspace{2cm}
    
    {\large Advisor: \advisorname \par}
    
    \vfill
    
    {\large Istanbul, \thesisdate \par}
\end{titlepage}

% Abstract
\chapter*{Abstract}
\addcontentsline{toc}{chapter}{Abstract}

Maritime vessel detection in the Bosphorus Strait is critical for traffic management, safety, and environmental monitoring. As one of the world's busiest waterways, the strait handles over 40,000 vessels annually, necessitating robust automated surveillance systems. However, general-purpose object detection models trained on standard datasets struggle with domain-specific maritime imagery due to unique lighting conditions, vessel types, and background complexity.

This thesis presents a comprehensive study of deep learning-based vessel detection for the Bosphorus Strait, focusing on the YOLO (You Only Look Once) family of object detectors. We demonstrate that vanilla COCO-pretrained YOLO models fail catastrophically on Bosphorus maritime imagery, achieving only 1--11\% recall despite including a ``boat'' class. Through domain-specific fine-tuning of YOLO11s with 581 training images, we achieve 87.54\% recall and 80.06\% precision (F1=0.836), representing an 8$\times$ improvement in detection performance.

The thesis provides an in-depth analysis of the YOLO architecture evolution from YOLOv1 to YOLO11, theoretical foundations of convolutional neural networks for object detection, and comprehensive evaluation methodology. We conduct ablation studies on input resolution, confidence thresholds, and training data size to quantify their impact on detection performance. Error analysis identifies primary failure modes including shore structure misdetection, overlapping vessels, and wake foam interference.

Our results demonstrate that domain-specific fine-tuning is essential, not optional, for maritime vessel detection applications. The fine-tuned model maintains real-time inference capability at 30+ FPS while achieving production-ready detection accuracy.

\vspace{1cm}
\textbf{Keywords:} YOLO, ship detection, maritime surveillance, transfer learning, object detection, Bosphorus Strait, deep learning, convolutional neural networks

% Özet (Turkish Abstract)
\chapter*{Özet}
\addcontentsline{toc}{chapter}{Özet}

İstanbul Boğazı'nda deniz aracı tespiti, trafik yönetimi, güvenlik ve çevre izleme açısından kritik öneme sahiptir. Dünyanın en işlek su yollarından biri olarak boğaz, yılda 40.000'den fazla gemiye ev sahipliği yapmakta ve güçlü otomatik gözetim sistemleri gerektirmektedir. Ancak standart veri setleri üzerinde eğitilmiş genel amaçlı nesne tespit modelleri, benzersiz aydınlatma koşulları, gemi türleri ve arka plan karmaşıklığı nedeniyle denizcilik görüntülerinde zorlanmaktadır.

Bu tez, YOLO (You Only Look Once) nesne dedektör ailesi üzerine odaklanarak, İstanbul Boğazı için derin öğrenme tabanlı gemi tespitinin kapsamlı bir çalışmasını sunmaktadır. COCO üzerinde önceden eğitilmiş vanilla YOLO modellerinin, ``tekne'' sınıfı içermesine rağmen Boğaz denizcilik görüntülerinde yalnızca \%1--11 geri çağırma oranı elde ederek başarısız olduğunu gösteriyoruz. 581 eğitim görüntüsü ile YOLO11s'in alana özgü ince ayarı sayesinde \%87.54 geri çağırma ve \%80.06 kesinlik (F1=0.836) elde ediyoruz; bu da tespit performansında 8 kat iyileşme anlamına gelmektedir.

Tez, YOLOv1'den YOLO11'e kadar YOLO mimari evriminin derinlemesine analizini, nesne tespiti için evrişimli sinir ağlarının teorik temellerini ve kapsamlı değerlendirme metodolojisini sunmaktadır. Giriş çözünürlüğü, güven eşikleri ve eğitim veri boyutunun tespit performansı üzerindeki etkisini ölçmek için ablasyon çalışmaları yürütüyoruz. Hata analizi, kıyı yapısı yanlış tespiti, üst üste binen gemiler ve dalga köpüğü paraziti dahil olmak üzere birincil başarısızlık modlarını tanımlar.

Sonuçlarımız, alana özgü ince ayarın deniz aracı tespit uygulamaları için isteğe bağlı değil, zorunlu olduğunu göstermektedir.

\vspace{1cm}
\textbf{Anahtar Kelimeler:} YOLO, gemi tespiti, denizcilik gözetimi, transfer öğrenme, nesne tespiti, İstanbul Boğazı, derin öğrenme, evrişimli sinir ağları

% Acknowledgements
\chapter*{Acknowledgements}
\addcontentsline{toc}{chapter}{Acknowledgements}

I would like to express my sincere gratitude to my thesis advisor, Dr. Rowanda D. Ahmed, for her continuous support, guidance, and encouragement throughout this research. Her expertise and insights have been invaluable in shaping this work.

I am grateful to Üsküdar University for providing the academic environment and resources necessary to complete this thesis. Special thanks to the faculty and staff of the Department of Computer Engineering for their support.

I would like to acknowledge:
\begin{itemize}
    \item Google Colab for providing GPU compute resources (NVIDIA L4)
    \item Ultralytics for the open-source YOLO implementation
    \item Roboflow for dataset hosting and annotation tools
    \item The Bosphorus Vision Project for creating and sharing the bogaz\_v\_1 dataset under CC BY 4.0 license
\end{itemize}

Finally, I extend my heartfelt thanks to my family and friends for their unwavering support and encouragement throughout my academic journey.

% Table of Contents
\tableofcontents

% List of Figures
\listoffigures
\addcontentsline{toc}{chapter}{List of Figures}

% List of Tables
\listoftables
\addcontentsline{toc}{chapter}{List of Tables}

% List of Abbreviations
\chapter*{List of Abbreviations}
\addcontentsline{toc}{chapter}{List of Abbreviations}

\begin{tabular}{ll}
    \textbf{AIS} & Automatic Identification System \\
    \textbf{AP} & Average Precision \\
    \textbf{AMP} & Automatic Mixed Precision \\
    \textbf{BCE} & Binary Cross-Entropy \\
    \textbf{CNN} & Convolutional Neural Network \\
    \textbf{COCO} & Common Objects in Context \\
    \textbf{C2PSA} & Cross Stage Partial with Spatial Attention \\
    \textbf{C3k2} & Cross Stage Partial with 2 Convolutions \\
    \textbf{DFL} & Distribution Focal Loss \\
    \textbf{FN} & False Negative \\
    \textbf{FP} & False Positive \\
    \textbf{FPN} & Feature Pyramid Network \\
    \textbf{FPS} & Frames Per Second \\
    \textbf{GFLOPs} & Giga Floating Point Operations \\
    \textbf{GPU} & Graphics Processing Unit \\
    \textbf{IoU} & Intersection over Union \\
    \textbf{mAP} & mean Average Precision \\
    \textbf{NMS} & Non-Maximum Suppression \\
    \textbf{PANet} & Path Aggregation Network \\
    \textbf{R-CNN} & Region-based Convolutional Neural Network \\
    \textbf{ReLU} & Rectified Linear Unit \\
    \textbf{RPN} & Region Proposal Network \\
    \textbf{SiLU} & Sigmoid Linear Unit \\
    \textbf{SPP} & Spatial Pyramid Pooling \\
    \textbf{SPPF} & Spatial Pyramid Pooling Fast \\
    \textbf{SSD} & Single Shot MultiBox Detector \\
    \textbf{TP} & True Positive \\
    \textbf{YOLO} & You Only Look Once \\
\end{tabular}

%% ----------------------------------------------------------------------------
%% MAIN MATTER
%% ----------------------------------------------------------------------------
\mainmatter
\pagenumbering{arabic}

% Include chapters
%% Chapter 1: Introduction
%% Target: 6-8 pages

\chapter{Introduction}
\label{ch:introduction}

%% ============================================================================
\section{Background and Motivation}
\label{sec:background}
%% ============================================================================

The Bosphorus Strait, known in Turkish as \textit{İstanbul Boğazı}, stands as one of the world's most strategically vital and geographically challenging maritime corridors. This narrow waterway connects the Black Sea to the Sea of Marmara and, through the Dardanelles, to the Aegean and Mediterranean Seas. With a length of approximately 31 kilometers (19 miles) and a minimum width of merely 700 meters at its narrowest point near Rumelihisarı, the Bosphorus presents a unique combination of high traffic density, complex geography, and critical infrastructure that demands sophisticated surveillance systems \citep{turkishstraits2023}.

\subsection{Strategic Importance of the Bosphorus}

The Bosphorus Strait serves as the sole maritime connection between the Black Sea and the world's oceans, making it indispensable for international trade. Countries bordering the Black Sea---including Russia, Ukraine, Romania, Bulgaria, and Georgia---depend entirely on this passage for their maritime commerce with the global market. The strategic significance extends beyond commercial interests to encompass energy security, as the strait serves as a major conduit for oil and natural gas exports from the Caspian and Black Sea regions to European and global markets.

Historical records and contemporary statistics underscore the strait's importance:
\begin{itemize}
    \item Over 40,000 vessels transit the Bosphorus annually, making it one of the busiest waterways globally
    \item Approximately 3 million barrels of oil pass through daily, representing a significant portion of global energy transport
    \item The strait hosts over 2,500 daily local ferry crossings, connecting the European and Asian sides of Istanbul
    \item Commercial vessel traffic includes cargo ships, oil tankers, chemical carriers, passenger vessels, and fishing boats
\end{itemize}

\subsection{Maritime Traffic Management Challenges}

The unique geography of the Bosphorus creates exceptional challenges for maritime traffic management. The strait features 12 sharp turns along its 31-kilometer length, with some requiring course changes of up to 80 degrees. These navigational constraints, combined with strong and variable currents (up to 7-8 knots during certain conditions), crosswinds from the adjacent landmasses, and the presence of two intercontinental suspension bridges, create a demanding environment for vessel navigation.

The Turkish Straits Vessel Traffic Service (VTS), operated by the Directorate General of Coastal Safety, employs radar systems, Automatic Identification System (AIS) receivers, and shore-based visual monitoring to manage traffic flow. However, traditional surveillance methods face several limitations:

\begin{enumerate}
    \item \textbf{Operator Fatigue:} Human operators monitoring camera feeds experience attention degradation over extended periods, potentially missing critical events
    \item \textbf{Scalability Constraints:} The number of camera feeds that human operators can effectively monitor simultaneously is inherently limited
    \item \textbf{Response Latency:} Manual identification and classification of vessels introduces delays in decision-making processes
    \item \textbf{AIS Limitations:} Smaller vessels and non-cooperative targets may not transmit AIS signals, creating blind spots in the surveillance picture
    \item \textbf{Adverse Conditions:} Low visibility conditions (fog, rain, night) challenge traditional visual monitoring
\end{enumerate}

\subsection{The Promise of Automated Visual Detection}

Computer vision and deep learning technologies offer a compelling solution to these challenges. Automated vessel detection systems can:
\begin{itemize}
    \item Continuously monitor multiple camera feeds without fatigue
    \item Provide real-time detection and alerting capabilities
    \item Complement radar and AIS systems with visual verification
    \item Detect non-cooperative vessels lacking AIS transponders
    \item Generate structured data for traffic analysis and pattern recognition
\end{itemize}

The YOLO (You Only Look Once) family of object detectors has emerged as the leading approach for real-time visual detection, offering an attractive balance between accuracy and speed suitable for maritime surveillance applications \citep{redmon2016yolo, ultralytics2024yolo11}. However, as this thesis demonstrates, the direct application of general-purpose pretrained models to specialized maritime environments yields inadequate results, necessitating domain-specific adaptation.

%% ============================================================================
\section{Problem Statement}
\label{sec:problem}
%% ============================================================================

Despite the availability of sophisticated object detection models pretrained on large-scale datasets, their direct application to maritime vessel detection in the Bosphorus Strait presents fundamental challenges that motivate this research.

\subsection{The Domain Gap Problem}

Modern object detection models, including the YOLO family, are typically pretrained on the Microsoft COCO (Common Objects in Context) dataset \citep{lin2014coco}, which contains 80 object categories including a ``boat'' class. This pretraining provides models with general visual features and detection capabilities. However, the COCO ``boat'' class predominantly features:
\begin{itemize}
    \item Recreational vessels (sailboats, kayaks, motorboats, canoes)
    \item Varied contexts (lakes, rivers, marinas, oceans)
    \item Diverse backgrounds (docks, beaches, open water)
    \item Typically close-range imagery with large relative object sizes
\end{itemize}

In contrast, Bosphorus maritime imagery presents fundamentally different characteristics:
\begin{itemize}
    \item Commercial vessels (cargo ships, oil tankers, container ships, ferries)
    \item Consistent maritime corridor context with urban shorelines
    \item Bridge infrastructure (Bosphorus Bridge, Fatih Sultan Mehmet Bridge, Yavuz Sultan Selim Bridge)
    \item Wide range of scales from nearby ferries to distant cargo ships (20 meters to 2 kilometers from camera)
    \item Distinctive lighting conditions unique to the strait's geography
\end{itemize}

This semantic and visual gap between the training domain (COCO) and the deployment domain (Bosphorus) severely degrades detection performance. As demonstrated in this thesis, this ``domain gap'' results in vanilla COCO-pretrained models achieving only 1--11\% recall on Bosphorus test imagery---effectively failing to detect 89--99\% of vessels.

\subsection{Quantifying the Performance Gap}

Preliminary experiments reveal the severity of this performance degradation:

\begin{quote}
\textit{COCO-pretrained YOLO models achieve only 1--11\% recall on Bosphorus maritime imagery, missing the vast majority of vessels despite including a ``boat'' class in their training data. This represents a catastrophic failure mode for safety-critical maritime surveillance applications.}
\end{quote}

The performance gap manifests in multiple failure modes:
\begin{enumerate}
    \item \textbf{False Negatives (Missed Vessels):} The dominant failure mode, where vessels present in the scene are not detected
    \item \textbf{False Positives (Phantom Detections):} Shore structures, bridges, and other background elements misclassified as vessels
    \item \textbf{Poor Localization:} When detection occurs, bounding boxes often poorly align with actual vessel boundaries
\end{enumerate}

\subsection{The Need for Domain-Specific Adaptation}

The fundamental insight motivating this research is that domain-specific fine-tuning is \textit{essential}, not optional, for effective maritime vessel detection. Transfer learning---leveraging pretrained weights while adapting to the target domain---offers a practical path forward that combines:
\begin{itemize}
    \item The feature extraction capabilities learned from large-scale pretraining
    \item Domain-specific knowledge acquired from targeted fine-tuning
    \item Practical data requirements achievable with limited annotation effort
\end{itemize}

This thesis systematically quantifies the performance improvement achievable through domain-specific fine-tuning and analyzes the factors contributing to this improvement.

%% ============================================================================
\section{Research Objectives}
\label{sec:objectives}
%% ============================================================================

This thesis addresses the challenge of maritime vessel detection in the Bosphorus Strait through the following research objectives:

\begin{enumerate}
    \item \textbf{Evaluate Baseline Performance:} Systematically assess the detection performance of vanilla COCO-pretrained YOLO models (YOLOv8n, YOLOv8s, YOLO11n, YOLO11s) on Bosphorus maritime imagery to establish baseline performance and quantify the domain gap.
    
    \item \textbf{Develop Fine-tuned Model:} Design and implement a domain-specific fine-tuned YOLO11s model optimized for Bosphorus vessel detection, incorporating appropriate training strategies, hyperparameter selection, and input resolution optimization.
    
    \item \textbf{Quantify Improvement:} Measure the performance improvement achieved through fine-tuning across multiple metrics including precision, recall, F1 score, IoU accuracy, and inference speed.
    
    \item \textbf{Analyze Detection Quality:} Evaluate not only detection rate but also localization accuracy, examining how well detected bounding boxes align with actual vessel boundaries.
    
    \item \textbf{Conduct Ablation Studies:} Investigate the contribution of key factors including input resolution, confidence thresholds, and training data size to overall detection performance.
    
    \item \textbf{Characterize Failure Modes:} Analyze false positives and false negatives to understand the remaining limitations and guide future improvements.
\end{enumerate}

%% ============================================================================
\section{Research Questions}
\label{sec:research_questions}
%% ============================================================================

This thesis addresses the following research questions:

\begin{description}
    \item[RQ1:] \textit{How significant is the performance gap between COCO-pretrained YOLO models and domain-specific fine-tuned models for Bosphorus vessel detection?}
    
    This question examines the quantitative difference in detection performance, measured by precision, recall, and F1 score, between off-the-shelf pretrained models and our fine-tuned approach.
    
    \item[RQ2:] \textit{What detection quality, measured by localization accuracy (IoU) and inference speed, can be achieved through domain-specific fine-tuning?}
    
    Beyond detection rate, this question investigates the quality of detections in terms of how precisely bounding boxes localize vessels and whether real-time processing requirements can be satisfied.
    
    \item[RQ3:] \textit{What are the primary failure modes and remaining limitations of the fine-tuned detection system?}
    
    This question explores the types of errors (false positives and false negatives) that persist after fine-tuning, categorizing their sources and identifying opportunities for future improvement.
    
    \item[RQ4:] \textit{How do key hyperparameters---specifically input resolution and confidence threshold---affect the precision-recall trade-off?}
    
    This question investigates the sensitivity of detection performance to configurable parameters, providing guidance for deployment in different operational contexts.
\end{description}

%% ============================================================================
\section{Contributions}
\label{sec:contributions}
%% ============================================================================

This thesis makes the following contributions to the field of maritime object detection:

\begin{enumerate}
    \item \textbf{First Systematic Bosphorus Evaluation:} We present the first documented systematic evaluation of YOLO11 models for vessel detection specifically in the Bosphorus Strait, demonstrating that vanilla COCO-pretrained models achieve only 1--11\% recall---a previously unquantified performance gap.
    
    \item \textbf{High-Performance Fine-tuned Model:} We develop a fine-tuned YOLO11s model achieving 87.54\% recall with 80.06\% precision (F1=0.836), representing an 8$\times$ improvement over vanilla models. This transforms an essentially non-functional detection system into a production-ready solution.
    
    \item \textbf{Comprehensive Comparative Analysis:} We provide rigorous comparison of five model configurations (four vanilla, one fine-tuned) on a standardized test set of 60 images containing 321 vessel instances, with consistent evaluation methodology enabling fair comparison.
    
    \item \textbf{Ablation Studies:} We conduct systematic ablation experiments examining the impact of input resolution, confidence thresholds, and training data size on detection performance, providing insights transferable to other domain-specific detection tasks.
    
    \item \textbf{Error Analysis Framework:} We develop a comprehensive error analysis categorizing false positives (shore structures, overlapping vessels, wake patterns) and false negatives (distant vessels, occlusion, unusual vessel types), identifying specific failure modes and their relative frequencies.
    
    \item \textbf{Reproducible Methodology:} We document our evaluation framework, including data splits (seed=42), matching algorithms, and metric calculations, supporting reproducible maritime detection research.
\end{enumerate}

%% ============================================================================
\section{Thesis Organization}
\label{sec:organization}
%% ============================================================================

The remainder of this thesis is organized as follows:

\textbf{Chapter~\ref{ch:literature_review}: Literature Review} provides comprehensive coverage of related work, including the evolution of YOLO object detectors from YOLOv1 through YOLO11, maritime object detection challenges and existing approaches, and transfer learning principles for domain adaptation.

\textbf{Chapter~\ref{ch:theoretical_background}: Theoretical Background} establishes the theoretical foundations underlying this research, including convolutional neural network fundamentals, detailed YOLO11 architecture analysis, loss functions for object detection, and evaluation metrics.

\textbf{Chapter~\ref{ch:methodology}: Dataset and Methodology} describes the Bosphorus maritime dataset, data preprocessing and augmentation strategies, model training configuration, and the evaluation framework used for comparative analysis.

\textbf{Chapter~\ref{ch:experiments}: Experiments and Results} presents the experimental findings, including main comparison results, ablation studies on resolution and confidence thresholds, and detailed error analysis of false positives and false negatives.

\textbf{Chapter~\ref{ch:discussion}: Discussion} interprets the experimental results, explaining why fine-tuning provides dramatic improvement, analyzing practical implications for maritime surveillance, and acknowledging limitations of the current approach.

\textbf{Chapter~\ref{ch:conclusion}: Conclusion and Future Work} summarizes the thesis contributions, key findings, and outlines directions for future research including multi-class classification, night-time detection, and geographic generalization.

\textbf{Appendices} provide supplementary material including dataset sample images, complete training logs, evaluation code, and per-image detection results.

%% Chapter 2: Literature Review
%% Target: 12-15 pages

\chapter{Literature Review}
\label{ch:literature_review}

This chapter provides a comprehensive review of the literature relevant to maritime vessel detection using deep learning. We begin with an overview of object detection evolution, followed by a detailed examination of the YOLO family of detectors, maritime-specific detection challenges and solutions, and finally transfer learning principles that enable domain adaptation.

%% ============================================================================
\section{Evolution of Object Detection}
\label{sec:detection_evolution}
%% ============================================================================

Object detection---the task of localizing and classifying objects within images---has undergone a transformative evolution over the past decade. This section traces the development from traditional computer vision methods to modern deep learning approaches.

\subsection{Traditional Methods}

Before the deep learning revolution, object detection relied primarily on hand-crafted features combined with classical machine learning classifiers.

\subsubsection{Histogram of Oriented Gradients (HOG)}

The Histogram of Oriented Gradients (HOG) descriptor, introduced by Dalal and Triggs for pedestrian detection, represents images through the distribution of gradient orientations in localized portions of the image. HOG features capture edge and shape information effectively but require careful parameter tuning and struggle with significant appearance variations, occlusions, and viewpoint changes.

\subsubsection{Deformable Parts Model (DPM)}

The Deformable Parts Model extended HOG by representing objects as collections of parts with spatial relationships. DPM achieved state-of-the-art performance on the PASCAL VOC benchmark and dominated object detection until the advent of deep learning. However, DPM's computational cost and sensitivity to part configurations limited its applicability to real-time scenarios.

\subsubsection{Sliding Window Approaches}

Traditional detection frameworks employed sliding window strategies, exhaustively evaluating classifiers at multiple positions and scales across the image. This brute-force approach, while thorough, incurs substantial computational overhead and struggles to achieve real-time performance even with efficient classifiers.

\subsubsection{Limitations Driving Deep Learning Adoption}

Traditional methods faced fundamental limitations that motivated the transition to deep learning:
\begin{itemize}
    \item \textbf{Feature Engineering Burden:} Hand-crafted features require domain expertise and may not capture all relevant visual patterns
    \item \textbf{Limited Generalization:} Features designed for specific domains (e.g., pedestrians) may not transfer to other object categories
    \item \textbf{Scalability Issues:} Performance degraded with increasing numbers of object classes
    \item \textbf{Computational Constraints:} Real-time detection remained challenging with sliding window approaches
\end{itemize}

\subsection{Two-Stage Detectors}

The introduction of deep learning to object detection began with two-stage detectors that decompose detection into region proposal and classification phases.

\subsubsection{R-CNN: Region-Based Convolutional Neural Networks}

Girshick et al.~\citep{girshick2014rcnn} introduced R-CNN (Regions with CNN features), which dramatically improved detection accuracy by leveraging deep convolutional networks for feature extraction. R-CNN operates in three stages:
\begin{enumerate}
    \item Generate approximately 2,000 region proposals using selective search
    \item Extract fixed-size feature vectors from each proposal using a CNN (typically AlexNet or VGG)
    \item Classify each region using SVM classifiers and refine bounding boxes via regression
\end{enumerate}

While R-CNN achieved significant accuracy improvements over traditional methods, its multi-stage pipeline and redundant CNN computations for overlapping proposals resulted in slow inference (approximately 47 seconds per image), precluding real-time applications.

\subsubsection{Fast R-CNN}

Girshick~\citep{girshick2015fastrcnn} addressed R-CNN's computational inefficiency with Fast R-CNN, which processes the entire image through the CNN once and extracts features for each proposal from the resulting feature map using ROI (Region of Interest) pooling. This architectural change reduced inference time significantly while maintaining accuracy. Fast R-CNN also unified the classification and bounding box regression into a single multi-task loss, simplifying training.

\subsubsection{Faster R-CNN and Region Proposal Networks}

Ren et al.~\citep{ren2015fasterrcnn} completed the evolution to end-to-end trainable detection with Faster R-CNN, which replaced selective search with a Region Proposal Network (RPN). The RPN shares convolutional features with the detection network and learns to propose regions likely to contain objects. This integration:
\begin{itemize}
    \item Eliminated the computational bottleneck of selective search
    \item Enabled end-to-end training of the complete detection pipeline
    \item Achieved near real-time speeds (5 FPS) while maintaining accuracy
\end{itemize}

Faster R-CNN established the two-stage detection paradigm that balanced accuracy and efficiency, influencing subsequent architectures.

\subsubsection{Feature Pyramid Networks}

Lin et al.~\citep{lin2017fpn} introduced Feature Pyramid Networks (FPN), which construct a multi-scale feature pyramid from a single-scale input by combining low-resolution, semantically strong features with high-resolution, semantically weak features through top-down pathways and lateral connections. FPN addresses the challenge of detecting objects at multiple scales---particularly small objects---and has become a standard component in modern detection architectures.

\subsection{One-Stage Detectors}

While two-stage detectors achieved high accuracy, their sequential region proposal and classification phases limited inference speed. One-stage detectors emerged as an alternative paradigm, directly predicting object locations and classes in a single forward pass.

\subsubsection{SSD: Single Shot MultiBox Detector}

Liu et al.~\citep{liu2016ssd} introduced SSD (Single Shot MultiBox Detector), which predicts objects at multiple feature map scales without explicit region proposal generation. Key innovations include:
\begin{itemize}
    \item \textbf{Multi-scale Feature Maps:} Detection occurs at multiple layers with different receptive fields, enabling detection of objects at various sizes
    \item \textbf{Default Boxes:} Predefined anchor boxes at each feature map location, with network predicting offsets and class probabilities
    \item \textbf{End-to-End Training:} Single-stage training with multi-task loss combining localization and classification
\end{itemize}

SSD achieved real-time speeds (59 FPS at 300×300 resolution) while approaching two-stage detector accuracy, demonstrating the viability of one-stage approaches.

\subsubsection{RetinaNet and Focal Loss}

Lin et al.~\citep{lin2017retinanet} addressed the class imbalance problem that historically limited one-stage detector accuracy. During training, the vast majority of candidate locations are background (easy negatives), which can overwhelm the loss and degrade learning of hard examples.

RetinaNet introduced Focal Loss, which down-weights the contribution of easy examples, focusing training on hard negatives:
\begin{equation}
    FL(p_t) = -\alpha_t (1 - p_t)^\gamma \log(p_t)
\end{equation}
where $p_t$ is the estimated probability for the ground truth class, $\alpha_t$ is a class balancing factor, and $\gamma$ is the focusing parameter (typically $\gamma = 2$). This simple modification enabled one-stage detectors to match or exceed two-stage detector accuracy.

\subsubsection{The YOLO Paradigm}

Among one-stage detectors, the YOLO (You Only Look Once) family deserves particular attention as the focus of this thesis. YOLO fundamentally reframes detection as a regression problem, predicting bounding boxes and class probabilities directly from image pixels in a single network evaluation. The following section provides detailed coverage of YOLO's evolution.

%% ============================================================================
\section{YOLO Architecture Evolution}
\label{sec:yolo_evolution}
%% ============================================================================

The YOLO (You Only Look Once) family of detectors has profoundly influenced real-time object detection, achieving an exceptional balance between speed and accuracy. This section traces YOLO's evolution from its 2016 introduction through the latest YOLO11 release.

\subsection{YOLOv1: Unified Detection}

Redmon et al.~\citep{redmon2016yolo} introduced YOLO in 2016, fundamentally reimagining object detection as a single regression problem. The core insight was that detection could be performed in one forward pass by dividing the image into a grid and having each grid cell predict bounding boxes and class probabilities simultaneously.

\textbf{Architecture:} YOLOv1 employed a convolutional network inspired by GoogLeNet, with 24 convolutional layers followed by 2 fully connected layers. The network processes the input image (448×448) to produce a 7×7 grid, with each cell predicting 2 bounding boxes and 20 class probabilities (for PASCAL VOC).

\textbf{Key Innovations:}
\begin{itemize}
    \item \textbf{Unified Architecture:} Single network predicts boxes and classes simultaneously
    \item \textbf{Global Reasoning:} Each prediction considers the entire image context
    \item \textbf{Real-Time Speed:} 45 FPS on standard hardware, enabling real-time applications
\end{itemize}

\textbf{Limitations:} YOLOv1 struggled with small objects (only 2 boxes per cell), closely spaced objects, and unusual aspect ratios due to its coarse grid structure.

\subsection{YOLOv2 (YOLO9000): Better, Faster, Stronger}

Redmon and Farhadi~\citep{redmon2017yolo9000} introduced YOLOv2 with substantial improvements in accuracy and speed, along with YOLO9000---a model capable of detecting over 9,000 object categories through joint training on detection and classification datasets.

\textbf{Key Innovations:}
\begin{itemize}
    \item \textbf{Batch Normalization:} Added to all convolutional layers, improving regularization and convergence
    \item \textbf{High-Resolution Classifier:} Pretraining at 448×448 before detection fine-tuning
    \item \textbf{Anchor Boxes:} K-means clustering on training data to determine optimal anchor box dimensions
    \item \textbf{Darknet-19:} New backbone with 19 convolutional layers and global average pooling
    \item \textbf{Multi-Scale Training:} Random input sizes during training for scale robustness
    \item \textbf{Passthrough Layer:} Concatenates high-resolution features with low-resolution features
\end{itemize}

YOLOv2 achieved 78.6\% mAP on PASCAL VOC 2007 at 40 FPS, significantly improving over YOLOv1's 63.4\%.

\subsection{YOLOv3: Multi-Scale Predictions}

Redmon and Farhadi~\citep{redmon2018yolov3} released YOLOv3 with focus on improved small object detection through multi-scale predictions.

\textbf{Key Innovations:}
\begin{itemize}
    \item \textbf{Darknet-53:} Deeper backbone with residual connections, comprising 53 convolutional layers
    \item \textbf{Multi-Scale Predictions:} Detection at three different scales (52×52, 26×26, 13×13 for 416×416 input)
    \item \textbf{Independent Class Predictions:} Sigmoid activation for multi-label classification rather than softmax
    \item \textbf{Nine Anchor Boxes:} Three boxes at each scale, totaling nine anchors
\end{itemize}

YOLOv3 achieved competitive performance with ResNet-based detectors while maintaining superior inference speed, establishing YOLO as a practical choice for real-world applications.

\subsection{YOLOv4: Bag of Freebies and Specials}

Bochkovskiy et al.~\citep{bochkovskiy2020yolov4} introduced YOLOv4 with extensive experimentation on training techniques and architectural modifications, categorized as ``Bag of Freebies'' (training improvements) and ``Bag of Specials'' (architectural enhancements).

\textbf{Backbone Innovations:}
\begin{itemize}
    \item \textbf{CSPDarknet53:} Cross Stage Partial connections reducing computation while maintaining capacity
    \item \textbf{Mish Activation:} Smooth, non-monotonic activation function $f(x) = x \cdot \tanh(\ln(1 + e^x))$
    \item \textbf{Dropblock Regularization:} Structured dropout for convolutional layers
\end{itemize}

\textbf{Neck Innovations:}
\begin{itemize}
    \item \textbf{SPP (Spatial Pyramid Pooling):} Multi-scale feature aggregation through parallel pooling at different scales
    \item \textbf{PANet (Path Aggregation Network):} Enhanced feature pyramid with bottom-up path augmentation
\end{itemize}

\textbf{Training Innovations:}
\begin{itemize}
    \item \textbf{Mosaic Augmentation:} Combining four training images, improving small object detection
    \item \textbf{Self-Adversarial Training:} Perturbing images to create challenging examples
    \item \textbf{CIoU Loss:} Complete IoU loss considering overlap, distance, and aspect ratio
\end{itemize}

YOLOv4 achieved 43.5\% AP on MS COCO at 65 FPS on Tesla V100, establishing new state-of-the-art for real-time detection.

\subsection{YOLOv5: Ultralytics Implementation}

Although not published as a formal research paper, YOLOv5 by Ultralytics~\citep{ultralytics2020yolov5} became widely adopted due to its PyTorch implementation, ease of use, and active maintenance.

\textbf{Key Features:}
\begin{itemize}
    \item \textbf{PyTorch Native:} Clean, well-documented PyTorch implementation
    \item \textbf{Auto-Anchor:} Automatic anchor box optimization for custom datasets
    \item \textbf{Focus Layer:} Space-to-depth transformation reducing computation in early layers
    \item \textbf{Multiple Model Sizes:} n, s, m, l, x variants for different speed-accuracy trade-offs
    \item \textbf{Comprehensive Tooling:} Training, validation, inference, export, and deployment pipelines
\end{itemize}

\subsection{YOLOv6: Industrial Applications}

Li et al.~\citep{li2022yolov6} from Meituan developed YOLOv6 targeting industrial deployment with efficient reparameterizable backbones.

\textbf{Key Innovations:}
\begin{itemize}
    \item \textbf{RepVGG Backbone:} Reparameterizable architecture enabling training-time complexity with inference-time efficiency
    \item \textbf{Efficient Decoupled Head:} Separate classification and localization branches
    \item \textbf{SimOTA Label Assignment:} Simplified optimal transport assignment for positive sample selection
\end{itemize}

\subsection{YOLOv7: Trainable Bag-of-Freebies}

Wang et al.~\citep{wang2023yolov7} introduced YOLOv7 with architectural innovations focusing on efficient layer aggregation and model scaling.

\textbf{Key Innovations:}
\begin{itemize}
    \item \textbf{E-ELAN (Extended Efficient Layer Aggregation Network):} Enhanced feature aggregation without destroying gradient paths
    \item \textbf{Model Scaling:} Compound scaling of depth and width while maintaining optimal structure
    \item \textbf{Planned Re-parameterized Convolution:} Reparameterization strategies preserving feature richness
    \item \textbf{Coarse-to-Fine Lead Head:} Auxiliary heads during training for improved supervision
\end{itemize}

YOLOv7 achieved 56.8\% AP on MS COCO at 161 FPS (V100), demonstrating continued improvements in the speed-accuracy frontier.

\subsection{YOLOv8: Anchor-Free Detection}

Ultralytics~\citep{ultralytics2023yolov8} released YOLOv8 as the successor to YOLOv5, incorporating lessons from YOLOv6, YOLOv7, and YOLOX.

\textbf{Key Innovations:}
\begin{itemize}
    \item \textbf{Anchor-Free Detection:} Eliminates anchor boxes, directly predicting object centers
    \item \textbf{Decoupled Head:} Separate branches for objectness, classification, and regression
    \item \textbf{Distribution Focal Loss (DFL):} Predicting bounding box distributions rather than fixed offsets
    \item \textbf{Mosaic Augmentation Enhancement:} Improved mosaic strategies with close\_mosaic parameter
    \item \textbf{Unified API:} Single codebase for detection, segmentation, classification, and pose estimation
\end{itemize}

\subsection{YOLOv9: Programmable Gradient Information}

Wang and Liao~\citep{wang2024yolov9} introduced YOLOv9 with novel architectural concepts for preserving gradient information.

\textbf{Key Innovations:}
\begin{itemize}
    \item \textbf{Programmable Gradient Information (PGI):} Architecture enabling network to ``program'' which information to propagate
    \item \textbf{GELAN (Generalized Efficient Layer Aggregation Network):} Flexible aggregation supporting diverse computational blocks
    \item \textbf{Reversible Functions:} Preserving information across network depth
\end{itemize}

\subsection{YOLOv10: NMS-Free Detection}

Wang et al.~\citep{wang2024yolov10} from Tsinghua University introduced YOLOv10 focusing on end-to-end detection efficiency.

\textbf{Key Innovations:}
\begin{itemize}
    \item \textbf{NMS-Free Training:} Consistent dual assignments eliminating post-processing Non-Maximum Suppression
    \item \textbf{Efficiency-Accuracy Driven Design:} Holistic optimization of all components
    \item \textbf{Large-Kernel Convolutions:} Expanded receptive fields in certain layers
    \item \textbf{Partial Self-Attention:} Selective attention in higher-resolution stages
\end{itemize}

\subsection{YOLO11: Latest Ultralytics Release}

Ultralytics~\citep{ultralytics2024yolo11} released YOLO11 in late 2024, representing the current state-of-the-art in the YOLO family.

\textbf{Key Innovations:}
\begin{itemize}
    \item \textbf{C3k2 Blocks:} Refined Cross Stage Partial blocks with two 3×3 convolutions for efficient feature extraction
    \item \textbf{C2PSA (Cross Stage Partial with Spatial Attention):} Integration of spatial attention mechanisms into feature aggregation
    \item \textbf{SPPF (Spatial Pyramid Pooling Fast):} Efficient multi-scale feature pooling with sequential max pooling
    \item \textbf{Improved Anchor-Free Detection:} Refined center-based prediction with DFL
    \item \textbf{Enhanced Training Strategies:} Improved augmentation pipelines and learning rate schedules
\end{itemize}

Table~\ref{tab:yolo_evolution} summarizes the evolution of YOLO architectures.

\begin{table}[!ht]
\centering
\caption{Summary of YOLO architecture evolution}
\label{tab:yolo_evolution}
\begin{tabular}{llll}
\toprule
\textbf{Version} & \textbf{Year} & \textbf{Key Innovation} & \textbf{Reference} \\
\midrule
YOLOv1 & 2016 & Unified real-time detection & \citet{redmon2016yolo} \\
YOLOv2 & 2017 & Batch norm, anchor boxes & \citet{redmon2017yolo9000} \\
YOLOv3 & 2018 & Multi-scale predictions & \citet{redmon2018yolov3} \\
YOLOv4 & 2020 & CSPDarknet, mosaic, PANet & \citet{bochkovskiy2020yolov4} \\
YOLOv5 & 2020 & PyTorch, Focus layer & \citet{ultralytics2020yolov5} \\
YOLOv6 & 2022 & RepVGG backbone & \citet{li2022yolov6} \\
YOLOv7 & 2023 & E-ELAN, compound scaling & \citet{wang2023yolov7} \\
YOLOv8 & 2023 & Anchor-free, decoupled head & \citet{ultralytics2023yolov8} \\
YOLOv9 & 2024 & PGI, GELAN & \citet{wang2024yolov9} \\
YOLOv10 & 2024 & NMS-free detection & \citet{wang2024yolov10} \\
YOLO11 & 2024 & C3k2, C2PSA, SPPF & \citet{ultralytics2024yolo11} \\
\bottomrule
\end{tabular}
\end{table}

%% ============================================================================
\section{Maritime Object Detection}
\label{sec:maritime_detection}
%% ============================================================================

Maritime object detection presents unique challenges distinct from general-purpose detection. This section examines these challenges, existing datasets, and prior approaches to maritime surveillance.

\subsection{Challenges in Maritime Environments}

Prasad et al.~\citep{prasad2017video} provide a comprehensive survey of maritime video processing challenges. Key difficulties include:

\subsubsection{Environmental Factors}
\begin{itemize}
    \item \textbf{Variable Lighting:} Maritime scenes experience dramatic lighting variations from sunrise through sunset, with midday glare, golden hour effects, and twilight conditions each presenting distinct visual characteristics
    \item \textbf{Weather Effects:} Fog, rain, and haze reduce visibility and contrast; sea state (calm to rough) affects water appearance and can obscure vessel hulls
    \item \textbf{Reflections and Glare:} Water surface reflections create bright spots and can obscure vessels or generate false patterns
    \item \textbf{Atmospheric Effects:} Haze increases with distance, reducing contrast for far vessels
\end{itemize}

\subsubsection{Object-Specific Challenges}
\begin{itemize}
    \item \textbf{Scale Variation:} Vessels range from small fishing boats to large container ships, and distance from camera creates enormous apparent size differences (e.g., 20m to 2km)
    \item \textbf{Vessel Diversity:} Commercial ships, tankers, ferries, fishing boats, yachts, military vessels, and coast guard boats all require detection despite visual differences
    \item \textbf{Orientation Variation:} Vessels present different appearances when viewed from bow, stern, or side
    \item \textbf{Occlusion:} Vessels may partially occlude each other in crowded waters
    \item \textbf{Wake and Foam:} Vessel wakes create distinctive but variable patterns that may aid or confuse detection
\end{itemize}

\subsubsection{Background Complexity}
\begin{itemize}
    \item \textbf{Shore Structures:} Buildings, piers, docks, and cranes near shorelines may resemble vessel structures
    \item \textbf{Bridge Infrastructure:} Bridge towers and supports can generate false positives
    \item \textbf{Water Texture:} Wave patterns and water surface variations create complex backgrounds
    \item \textbf{Horizon Line:} The boundary between sea and sky varies with conditions
\end{itemize}

\subsection{Maritime Detection Datasets}

Several datasets have been developed for maritime object detection research:

\subsubsection{SeaShips Dataset}
Shao et al.~\citep{shao2018seaships} introduced the SeaShips dataset, comprising 31,455 images with 40,077 vessel instances across 6 categories (ore carrier, bulk cargo carrier, general cargo ship, container ship, passenger ship, and fishing boat). The dataset was captured from coastal surveillance cameras and provides precise bounding box annotations.

\subsubsection{Singapore Maritime Dataset (SMD)}
The Singapore Maritime Dataset~\citep{prasad2017video} contains over 81,000 frames from onshore and onboard cameras in Singapore waters. It includes annotations for 4 categories: ferry, buoy, vessel, and kayak. The dataset emphasizes near-shore vessel detection with challenging lighting and weather variations.

\subsubsection{ABOShips Dataset}
The ABOShips dataset from Åbo Akademi University contains approximately 9,880 images from the Baltic Sea region with 9 vessel categories. It captures the distinct vessel types and conditions of Northern European waters.

\subsubsection{Bosphorus Dataset}
The bogaz\_v\_1 dataset~\citep{roboflow2024bogaz} used in this thesis contains 859 images from the Bosphorus Strait with 4,859 vessel instances. While smaller than other datasets, it specifically captures the unique characteristics of Bosphorus maritime traffic, including distinctive vessel types, shore structures, and lighting conditions.

Table~\ref{tab:maritime_datasets} compares these maritime detection datasets.

\begin{table}[!ht]
\centering
\caption{Comparison of maritime detection datasets}
\label{tab:maritime_datasets}
\begin{tabular}{lcccc}
\toprule
\textbf{Dataset} & \textbf{Images} & \textbf{Instances} & \textbf{Classes} & \textbf{Region} \\
\midrule
SeaShips & 31,455 & 40,077 & 6 & China coast \\
SMD & 81,000+ & N/A & 4 & Singapore \\
ABOShips & 9,880 & N/A & 9 & Baltic Sea \\
Bosphorus & 859 & 4,859 & 1 & Bosphorus Strait \\
\bottomrule
\end{tabular}
\end{table}

\subsection{Prior Maritime Detection Approaches}

Maritime detection has traditionally relied on multiple sensor modalities:

\subsubsection{Radar-Based Detection}
Maritime radar systems detect vessels through electromagnetic reflection, providing range and bearing information regardless of lighting conditions. However, radar struggles with small vessels, has limited classification capability, and cannot provide visual verification.

\subsubsection{AIS-Based Tracking}
The Automatic Identification System (AIS) requires vessels to broadcast identification, position, and navigation data. While effective for cooperative vessels, AIS has limitations:
\begin{itemize}
    \item Small vessels may not carry AIS transponders
    \item Non-cooperative vessels may disable or spoof AIS
    \item AIS provides no visual confirmation
\end{itemize}

\subsubsection{Camera-Based Detection}
Camera-based detection using deep learning has emerged as a complement to radar and AIS:
\begin{itemize}
    \item Provides visual verification of vessel presence
    \item Can detect non-cooperative vessels
    \item Enables vessel classification from visual features
    \item Supports surveillance in areas with radar limitations
\end{itemize}

\subsubsection{Deep Learning for Maritime Detection}
Recent work has applied CNN-based detectors to maritime imagery. Kanjir et al.~\citep{kanjir2018vessel} survey vessel detection from satellite imagery, while shore-based detection typically employs architectures from the YOLO, Faster R-CNN, or SSD families. Key findings from prior work include:
\begin{itemize}
    \item Domain-specific training significantly outperforms general pretrained models
    \item Multi-scale detection is essential for handling vessel size variation
    \item Data augmentation helps with limited maritime training data
    \item False positives from shore structures remain a challenge
\end{itemize}

%% ============================================================================
\section{Transfer Learning for Domain Adaptation}
\label{sec:transfer_learning}
%% ============================================================================

Transfer learning enables models trained on large source datasets to adapt efficiently to target domains with limited data. This section examines transfer learning principles relevant to maritime detection.

\subsection{Transfer Learning Theory}

Pan and Yang~\citep{pan2010survey} provide the foundational survey on transfer learning. The key insight is that knowledge learned from solving one task can be transferred to improve learning on a related but different task.

\subsubsection{Problem Formulation}
Given a source domain $\mathcal{D}_S$ with learning task $\mathcal{T}_S$ and a target domain $\mathcal{D}_T$ with learning task $\mathcal{T}_T$, transfer learning aims to improve the learning of the target predictive function $f_T(\cdot)$ using knowledge from $\mathcal{D}_S$ and $\mathcal{T}_S$, where $\mathcal{D}_S \neq \mathcal{D}_T$ or $\mathcal{T}_S \neq \mathcal{T}_T$.

\subsubsection{Types of Transfer Learning}
\begin{itemize}
    \item \textbf{Inductive Transfer:} Source and target tasks differ; labeled data in target domain
    \item \textbf{Transductive Transfer:} Tasks are the same; domains differ (domain adaptation)
    \item \textbf{Unsupervised Transfer:} No labeled data in source or target domains
\end{itemize}

For object detection, domain adaptation (transductive transfer) is most relevant, as the task (detection) remains the same but the domain (COCO $\rightarrow$ Bosphorus) differs.

\subsection{Transfer Learning in Object Detection}

Weiss et al.~\citep{weiss2016survey} extend the transfer learning survey to deep learning contexts. For object detection, transfer learning typically involves:

\subsubsection{Pretrained Backbones}
Detection networks use backbones (ResNet, Darknet, CSPDarknet) pretrained on large classification datasets (ImageNet~\citep{deng2009imagenet}) or detection datasets (COCO~\citep{lin2014coco}). Pretrained weights provide:
\begin{itemize}
    \item Effective low-level feature extractors (edges, textures, shapes)
    \item Mid-level semantic features (object parts, patterns)
    \item Initialization that avoids poor local minima
\end{itemize}

\subsubsection{Fine-Tuning Strategies}
Common fine-tuning approaches include:
\begin{itemize}
    \item \textbf{Full Fine-Tuning:} Update all network weights with reduced learning rate
    \item \textbf{Feature Extraction:} Freeze backbone, only train detection head
    \item \textbf{Gradual Unfreezing:} Progressively unfreeze layers from head to backbone
    \item \textbf{Discriminative Learning Rates:} Lower learning rates for early layers, higher for later layers
\end{itemize}

\subsubsection{Head Replacement}
When target domain has different classes than source, the classification head must be replaced:
\begin{itemize}
    \item COCO: 80 classes including ``boat''
    \item Bosphorus: 1 class (``gemiler''/ships)
\end{itemize}
The new head is randomly initialized while backbone weights are transferred.

\subsection{Domain Gap in Maritime Detection}

The domain gap between COCO and Bosphorus imagery manifests in multiple dimensions:

\subsubsection{Semantic Gap}
COCO ``boat'' predominantly features recreational vessels (sailboats, kayaks, motorboats) in varied contexts (lakes, rivers, marinas). Bosphorus traffic comprises commercial vessels (cargo ships, tankers, ferries) in a consistent maritime corridor context.

\subsubsection{Visual Gap}
\begin{itemize}
    \item \textbf{Scale:} COCO boats typically appear large in frame; Bosphorus vessels range from close to very distant
    \item \textbf{Background:} COCO has diverse backgrounds; Bosphorus has consistent maritime/urban backdrop
    \item \textbf{Lighting:} Bosphorus-specific lighting conditions differ from COCO's variety
\end{itemize}

\subsubsection{Quantifying Domain Gap}
This thesis quantifies the domain gap by comparing vanilla COCO-pretrained models (1--11\% recall) against fine-tuned models (87.54\% recall). This 8$\times$ improvement demonstrates the severity of the domain gap and the necessity of domain adaptation.

\subsection{When Transfer Learning Helps}

Transfer learning effectiveness depends on source-target similarity:
\begin{itemize}
    \item \textbf{Positive Transfer:} Source knowledge improves target performance (COCO $\rightarrow$ Bosphorus with fine-tuning)
    \item \textbf{Negative Transfer:} Source knowledge degrades target performance (rare with appropriate fine-tuning)
    \item \textbf{No Transfer:} Source provides no benefit (training from scratch)
\end{itemize}

For Bosphorus maritime detection, COCO pretraining provides positive transfer through:
\begin{itemize}
    \item General object detection features (edges, shapes, textures)
    \item Boat-like feature representations (despite semantic differences)
    \item Optimized network architecture and training recipes
\end{itemize}

However, direct application without fine-tuning results in near-complete failure, demonstrating that pretraining alone is insufficient for domain-specific deployment.

%% ============================================================================
\section{Summary}
\label{sec:literature_summary}
%% ============================================================================

This chapter has reviewed the literature foundational to maritime vessel detection using deep learning:

\begin{enumerate}
    \item \textbf{Object Detection Evolution:} From traditional HOG/DPM methods through two-stage detectors (R-CNN family) to one-stage detectors (SSD, RetinaNet, YOLO), achieving real-time performance with high accuracy.
    
    \item \textbf{YOLO Architecture Evolution:} From YOLOv1's unified detection paradigm through eleven major versions, each introducing architectural and training innovations that advance the speed-accuracy frontier.
    
    \item \textbf{Maritime Detection Challenges:} Unique challenges including environmental factors, vessel diversity, scale variation, and background complexity that distinguish maritime detection from general object detection.
    
    \item \textbf{Transfer Learning:} Principles enabling efficient domain adaptation from large-scale pretraining to specialized deployment domains, essential for maritime applications with limited training data.
\end{enumerate}

The literature establishes that while YOLO detectors achieve excellent general object detection performance, domain-specific adaptation is essential for specialized applications like Bosphorus maritime surveillance. The following chapters detail our methodology for fine-tuning YOLO11s and present comprehensive experimental results quantifying the improvement achieved.

%% Chapter 3: Theoretical Background
%% Target: 10-12 pages

\chapter{Theoretical Background}
\label{ch:theoretical_background}

This chapter establishes the theoretical foundations underlying deep learning-based object detection. We begin with convolutional neural network fundamentals, proceed to a detailed analysis of the YOLO11 architecture used in this thesis, examine the loss functions that guide training, and conclude with formal definitions of evaluation metrics.

%% ============================================================================
\section{Convolutional Neural Networks}
\label{sec:cnn_fundamentals}
%% ============================================================================

Convolutional Neural Networks (CNNs) form the backbone of modern computer vision systems, including object detectors. This section reviews the fundamental components and principles that enable CNNs to extract meaningful features from images.

\subsection{The Convolution Operation}

The core operation in CNNs is the discrete convolution between an input tensor and a learnable filter (kernel). For a 2D input image $I$ and a kernel $K$ of size $k \times k$, the convolution at position $(i, j)$ is:

\begin{equation}
    (I * K)(i, j) = \sum_{m=0}^{k-1} \sum_{n=0}^{k-1} I(i+m, j+n) \cdot K(m, n)
\end{equation}

For multi-channel inputs (e.g., RGB images with 3 channels), the convolution operates across all input channels and sums the results:

\begin{equation}
    (I * K)(i, j) = \sum_{c=0}^{C_{in}-1} \sum_{m=0}^{k-1} \sum_{n=0}^{k-1} I(i+m, j+n, c) \cdot K(m, n, c)
\end{equation}

where $C_{in}$ is the number of input channels. A convolutional layer typically applies multiple filters to produce multiple output channels (feature maps).

\subsubsection{Stride and Padding}

Two key parameters control the convolution output dimensions:

\begin{itemize}
    \item \textbf{Stride ($s$):} The step size between consecutive convolution positions. Stride $> 1$ reduces spatial dimensions.
    \item \textbf{Padding ($p$):} Zero-values added around the input border. Padding preserves spatial dimensions or controls output size.
\end{itemize}

Given input size $H_{in} \times W_{in}$, kernel size $k$, stride $s$, and padding $p$, the output dimensions are:

\begin{equation}
    H_{out} = \left\lfloor \frac{H_{in} + 2p - k}{s} \right\rfloor + 1, \quad W_{out} = \left\lfloor \frac{W_{in} + 2p - k}{s} \right\rfloor + 1
\end{equation}

\subsubsection{Feature Maps and Hierarchical Features}

The output of a convolutional layer is called a \textit{feature map}. Successive convolutional layers build a hierarchy of features:

\begin{enumerate}
    \item \textbf{Early Layers:} Detect low-level features (edges, corners, textures)
    \item \textbf{Middle Layers:} Combine low-level features into mid-level patterns (shapes, parts)
    \item \textbf{Deep Layers:} Represent high-level semantic concepts (objects, scenes)
\end{enumerate}

This hierarchical feature extraction is fundamental to CNNs' ability to recognize objects regardless of position, scale, and minor variations.

\subsection{Pooling Layers}

Pooling layers reduce spatial dimensions while retaining important features, providing translation invariance and reducing computational cost.

\subsubsection{Max Pooling}

Max pooling selects the maximum value within each pooling window:

\begin{equation}
    \text{MaxPool}(i, j) = \max_{(m, n) \in \mathcal{R}_{ij}} I(m, n)
\end{equation}

where $\mathcal{R}_{ij}$ is the pooling region at position $(i, j)$. Max pooling preserves the strongest activation, which typically corresponds to the presence of the feature.

\subsubsection{Average Pooling}

Average pooling computes the mean value within each pooling window:

\begin{equation}
    \text{AvgPool}(i, j) = \frac{1}{|\mathcal{R}_{ij}|} \sum_{(m, n) \in \mathcal{R}_{ij}} I(m, n)
\end{equation}

Average pooling provides smoother downsampling but may dilute strong activations.

\subsubsection{Global Average Pooling}

Global Average Pooling (GAP) computes the average across the entire spatial extent of each channel, producing a single value per channel. GAP is commonly used before the final classification layer, reducing spatial dimensions to 1×1.

\subsection{Activation Functions}

Activation functions introduce non-linearity, enabling networks to learn complex mappings.

\subsubsection{ReLU (Rectified Linear Unit)}

The Rectified Linear Unit~\citep{krizhevsky2017imagenet} is the most widely used activation:

\begin{equation}
    \text{ReLU}(x) = \max(0, x)
\end{equation}

ReLU is computationally efficient and avoids the vanishing gradient problem for positive values. However, ``dying ReLU'' can occur when units become permanently inactive (always outputting zero).

\subsubsection{Leaky ReLU}

Leaky ReLU addresses dying ReLU by allowing small negative values:

\begin{equation}
    \text{LeakyReLU}(x) = \begin{cases} x & \text{if } x > 0 \\ \alpha x & \text{if } x \leq 0 \end{cases}
\end{equation}

where $\alpha$ is a small constant (typically 0.01 or 0.1).

\subsubsection{SiLU (Sigmoid Linear Unit)}

SiLU, also known as Swish, is used in modern YOLO architectures:

\begin{equation}
    \text{SiLU}(x) = x \cdot \sigma(x) = \frac{x}{1 + e^{-x}}
\end{equation}

where $\sigma(x)$ is the sigmoid function. SiLU is smooth, non-monotonic, and has been shown to improve training in deep networks.

\subsubsection{Mish Activation}

Mish, used in YOLOv4, is another smooth activation:

\begin{equation}
    \text{Mish}(x) = x \cdot \tanh(\text{softplus}(x)) = x \cdot \tanh(\ln(1 + e^x))
\end{equation}

\subsection{Batch Normalization}

Batch Normalization~\citep{ioffe2015batchnorm} normalizes layer inputs across the mini-batch, stabilizing training and enabling higher learning rates.

For a mini-batch $\mathcal{B} = \{x_1, \ldots, x_m\}$, batch normalization computes:

\begin{equation}
    \mu_{\mathcal{B}} = \frac{1}{m} \sum_{i=1}^{m} x_i, \quad \sigma_{\mathcal{B}}^2 = \frac{1}{m} \sum_{i=1}^{m} (x_i - \mu_{\mathcal{B}})^2
\end{equation}

\begin{equation}
    \hat{x}_i = \frac{x_i - \mu_{\mathcal{B}}}{\sqrt{\sigma_{\mathcal{B}}^2 + \epsilon}}
\end{equation}

\begin{equation}
    y_i = \gamma \hat{x}_i + \beta
\end{equation}

where $\gamma$ and $\beta$ are learnable parameters that allow the network to undo the normalization if beneficial, and $\epsilon$ is a small constant for numerical stability.

\subsection{Residual Connections}

He et al.~\citep{he2016resnet} introduced residual connections to enable training of very deep networks by providing shortcut paths for gradient flow.

A residual block computes:

\begin{equation}
    \mathbf{y} = \mathcal{F}(\mathbf{x}, \{W_i\}) + \mathbf{x}
\end{equation}

where $\mathcal{F}(\mathbf{x}, \{W_i\})$ is the residual mapping (typically two or three convolutional layers) and $\mathbf{x}$ is the identity shortcut. The key insight is that learning the residual $\mathcal{F}(\mathbf{x})$ is easier than learning the full mapping when the desired output is close to the input.

Residual connections enable:
\begin{itemize}
    \item Training of networks with hundreds of layers
    \item Improved gradient flow during backpropagation
    \item Better feature reuse across layers
\end{itemize}

\subsection{Depthwise Separable Convolutions}

Depthwise separable convolutions reduce computational cost by decomposing standard convolution into:

\begin{enumerate}
    \item \textbf{Depthwise Convolution:} Apply a single filter per input channel
    \item \textbf{Pointwise Convolution:} 1×1 convolution to combine channel outputs
\end{enumerate}

For input with $C_{in}$ channels, kernel size $k$, and $C_{out}$ output channels:
\begin{itemize}
    \item Standard convolution: $k^2 \cdot C_{in} \cdot C_{out}$ multiplications per output position
    \item Depthwise separable: $k^2 \cdot C_{in} + C_{in} \cdot C_{out}$ multiplications per output position
\end{itemize}

The reduction factor is approximately $\frac{1}{C_{out}} + \frac{1}{k^2}$, which for typical values ($C_{out} = 256$, $k = 3$) is roughly 8-9×.

%% ============================================================================
\section{YOLO11 Architecture}
\label{sec:yolo11_architecture}
%% ============================================================================

This section provides a detailed analysis of the YOLO11 architecture, specifically the YOLO11s (small) variant used in this thesis. YOLO11 builds upon previous YOLO versions with refined architectural components for improved efficiency and accuracy.

\subsection{Overall Architecture}

YOLO11 follows the encoder-decoder paradigm common to modern object detectors:

\begin{enumerate}
    \item \textbf{Backbone:} Extracts hierarchical features from input images
    \item \textbf{Neck:} Aggregates multi-scale features from the backbone
    \item \textbf{Head:} Produces detection outputs (bounding boxes and class probabilities)
\end{enumerate}

Figure~\ref{fig:yolo11_architecture} illustrates the high-level architecture.

\begin{figure}[!ht]
\centering
\fbox{\parbox{0.9\textwidth}{
\centering
\textbf{YOLO11 Architecture Overview}\\[10pt]
\begin{tabular}{c}
Input Image (1088×1088×3) \\
$\downarrow$ \\
\textbf{Backbone} (C3k2, SPPF) \\
$\downarrow$ Feature Maps: P3, P4, P5 \\
$\downarrow$ \\
\textbf{Neck} (PANet + C2PSA) \\
$\downarrow$ Multi-scale Features \\
$\downarrow$ \\
\textbf{Detection Heads} (P3, P4, P5) \\
$\downarrow$ \\
Predictions + NMS $\rightarrow$ Final Detections
\end{tabular}
}}
\caption{High-level YOLO11 architecture showing backbone, neck, and detection head components}
\label{fig:yolo11_architecture}
\end{figure}

\subsection{Backbone Network}

The YOLO11 backbone extracts features through a series of convolutional blocks with increasing receptive fields and channel dimensions.

\subsubsection{Conv Blocks}

The basic building block is the Conv module, which combines:
\begin{enumerate}
    \item Convolution (3×3 or 1×1)
    \item Batch Normalization
    \item SiLU Activation
\end{enumerate}

\subsubsection{C3k2 Blocks}

C3k2 (Cross Stage Partial with 2 convolutions) is the primary feature extraction module in YOLO11. It extends the CSP (Cross Stage Partial) concept with efficient 3×3 convolutions.

The C3k2 block:
\begin{enumerate}
    \item Splits input into two branches
    \item Processes one branch through a sequence of bottleneck modules
    \item Concatenates both branches
    \item Applies a final convolution to fuse features
\end{enumerate}

This design:
\begin{itemize}
    \item Reduces computational cost by processing only half the channels through expensive operations
    \item Maintains gradient flow through the direct path
    \item Enables rich feature combination through concatenation
\end{itemize}

\subsubsection{SPPF (Spatial Pyramid Pooling Fast)}

SPPF aggregates features at multiple scales through sequential max pooling operations. Unlike the original SPP with parallel pooling at different kernel sizes, SPPF applies sequential 5×5 max pooling operations:

\begin{equation}
    \text{SPPF}(x) = \text{Conv}([\text{x}, \text{MaxPool}(x), \text{MaxPool}^2(x), \text{MaxPool}^3(x)])
\end{equation}

where $[\cdot]$ denotes concatenation and $\text{MaxPool}^n$ indicates $n$ sequential applications. This achieves the same receptive field expansion as parallel pooling with reduced computation.

\subsection{Neck: Feature Aggregation}

The neck combines features from different backbone levels to enable detection at multiple scales.

\subsubsection{Path Aggregation Network (PANet)}

PANet enhances the standard top-down Feature Pyramid Network (FPN) with an additional bottom-up pathway, creating bidirectional feature flow:

\begin{enumerate}
    \item \textbf{Top-Down Path:} High-level semantic features flow to lower levels
    \item \textbf{Bottom-Up Path:} High-resolution spatial features flow to higher levels
\end{enumerate}

This bidirectional aggregation ensures that each detection scale has access to both semantic and spatial information.

\subsubsection{C2PSA (Cross Stage Partial with Spatial Attention)}

C2PSA modules incorporate spatial attention mechanisms into the feature aggregation:

\begin{equation}
    \text{C2PSA}(x) = x + \text{SA}(\text{CSP}(x))
\end{equation}

where SA denotes spatial attention. Spatial attention helps the network focus on relevant image regions, suppressing background noise and emphasizing object locations.

\subsection{Detection Head}

YOLO11 uses anchor-free detection with decoupled heads for classification and localization.

\subsubsection{Multi-Scale Detection}

Detection occurs at three scales corresponding to different feature map resolutions:

\begin{itemize}
    \item \textbf{P3 (Large Scale):} High resolution, detects small objects
    \item \textbf{P4 (Medium Scale):} Medium resolution, detects medium objects
    \item \textbf{P5 (Small Scale):} Low resolution, detects large objects
\end{itemize}

For input size 1088×1088, the feature map sizes are approximately:
\begin{itemize}
    \item P3: 136×136
    \item P4: 68×68
    \item P5: 34×34
\end{itemize}

\subsubsection{Decoupled Head}

Unlike coupled heads that share features for classification and localization, YOLO11 uses separate branches:

\begin{itemize}
    \item \textbf{Classification Branch:} Predicts class probabilities using sigmoid activation
    \item \textbf{Regression Branch:} Predicts bounding box coordinates using Distribution Focal Loss
\end{itemize}

Decoupling allows each task to optimize its representations independently.

\subsubsection{Anchor-Free Detection}

YOLO11 eliminates predefined anchor boxes, instead predicting:
\begin{itemize}
    \item Object center location $(x, y)$ relative to grid cell
    \item Bounding box dimensions $(w, h)$ using DFL
\end{itemize}

Anchor-free detection simplifies the pipeline and improves generalization to objects with unusual aspect ratios.

\subsection{YOLO11s Model Specifications}

Table~\ref{tab:yolo11s_specs} details the YOLO11s variant used in this thesis.

\begin{table}[!ht]
\centering
\caption{YOLO11s model specifications}
\label{tab:yolo11s_specs}
\begin{tabular}{lc}
\toprule
\textbf{Property} & \textbf{Value} \\
\midrule
Total layers & 181 \\
Layers after fusion & 100 \\
Parameters & 9,428,179 (9.4M) \\
GFLOPs & 21.5 \\
Input resolution (thesis) & 1088×1088 \\
Detection scales & 3 (P3, P4, P5) \\
Head channels & 128, 256, 512 \\
Anchor-free & Yes \\
Activation & SiLU \\
Normalization & Batch Normalization \\
\bottomrule
\end{tabular}
\end{table}

%% ============================================================================
\section{Loss Functions for Object Detection}
\label{sec:loss_functions}
%% ============================================================================

Training object detectors requires loss functions that capture both localization accuracy and classification correctness. YOLO11 employs a composite loss combining multiple components.

\subsection{Box Regression Loss}

Box regression loss measures the discrepancy between predicted and ground truth bounding boxes.

\subsubsection{IoU-Based Losses}

Intersection over Union (IoU) directly measures overlap between predicted box $B_p$ and ground truth box $B_{gt}$:

\begin{equation}
    \text{IoU} = \frac{|B_p \cap B_{gt}|}{|B_p \cup B_{gt}|}
\end{equation}

IoU loss is simply $\mathcal{L}_{IoU} = 1 - \text{IoU}$.

\subsubsection{GIoU (Generalized IoU)}

GIoU addresses cases where boxes don't overlap by considering the smallest enclosing box $B_c$:

\begin{equation}
    \text{GIoU} = \text{IoU} - \frac{|B_c \setminus (B_p \cup B_{gt})|}{|B_c|}
\end{equation}

GIoU ranges from -1 to 1, with higher values indicating better overlap.

\subsubsection{DIoU (Distance IoU)}

DIoU adds a penalty for center distance:

\begin{equation}
    \text{DIoU} = \text{IoU} - \frac{\rho^2(\mathbf{b}_p, \mathbf{b}_{gt})}{c^2}
\end{equation}

where $\rho$ is Euclidean distance between box centers and $c$ is the diagonal length of the enclosing box.

\subsubsection{CIoU (Complete IoU)}

CIoU further considers aspect ratio consistency:

\begin{equation}
    \text{CIoU} = \text{IoU} - \frac{\rho^2(\mathbf{b}_p, \mathbf{b}_{gt})}{c^2} - \alpha v
\end{equation}

where:
\begin{equation}
    v = \frac{4}{\pi^2}\left(\arctan\frac{w_{gt}}{h_{gt}} - \arctan\frac{w_p}{h_p}\right)^2
\end{equation}

and $\alpha$ is a weighting factor. CIoU provides comprehensive guidance for box refinement.

\subsubsection{Distribution Focal Loss (DFL)}

YOLO11 uses Distribution Focal Loss for bounding box regression, predicting a discrete probability distribution over possible coordinate values rather than point estimates. For coordinate $y$ with label $y_i < y < y_{i+1}$:

\begin{equation}
    \mathcal{L}_{DFL} = -\left((y_{i+1} - y)\log(P(y_i)) + (y - y_i)\log(P(y_{i+1}))\right)
\end{equation}

DFL enables the network to express uncertainty in predictions and has been shown to improve localization accuracy.

\subsection{Classification Loss}

Classification loss measures the correctness of class predictions.

\subsubsection{Binary Cross-Entropy (BCE)}

For multi-label classification (objects can belong to multiple classes), BCE is applied independently per class:

\begin{equation}
    \mathcal{L}_{BCE} = -\sum_{c=1}^{C}\left[y_c \log(\hat{y}_c) + (1-y_c)\log(1-\hat{y}_c)\right]
\end{equation}

where $y_c \in \{0, 1\}$ is the ground truth label for class $c$ and $\hat{y}_c$ is the predicted probability.

\subsubsection{Focal Loss}

Focal Loss~\citep{lin2017retinanet} addresses class imbalance by down-weighting easy examples:

\begin{equation}
    \mathcal{L}_{FL} = -\alpha_t (1 - p_t)^\gamma \log(p_t)
\end{equation}

where $p_t$ is the probability of the correct class, $\alpha_t$ balances positive/negative examples, and $\gamma$ (typically 2) controls the focusing strength.

\subsection{Objectness Loss}

Objectness loss measures whether an object exists at a given location. In anchor-free detection, this is typically BCE applied to the objectness score, with targets assigned based on center distance to ground truth objects.

\subsection{Total Loss}

The total YOLO11 loss combines all components:

\begin{equation}
    \mathcal{L}_{total} = \lambda_{box} \mathcal{L}_{box} + \lambda_{cls} \mathcal{L}_{cls} + \lambda_{dfl} \mathcal{L}_{dfl}
\end{equation}

where $\lambda_{box}$, $\lambda_{cls}$, and $\lambda_{dfl}$ are weighting hyperparameters balancing the contribution of each loss component.

%% ============================================================================
\section{Evaluation Metrics}
\label{sec:evaluation_metrics}
%% ============================================================================

Rigorous evaluation requires well-defined metrics~\citep{padilla2020survey}. This section formally defines the metrics used throughout this thesis.

\subsection{Intersection over Union (IoU)}

IoU measures the overlap between predicted bounding box $B_p$ and ground truth bounding box $B_{gt}$:

\begin{equation}
    \text{IoU}(B_p, B_{gt}) = \frac{\text{Area}(B_p \cap B_{gt})}{\text{Area}(B_p \cup B_{gt})}
\end{equation}

IoU ranges from 0 (no overlap) to 1 (perfect overlap). A detection is typically considered correct if IoU $\geq$ 0.5 (COCO standard).

\subsection{True Positives, False Positives, False Negatives}

For a given IoU threshold $\tau$:

\begin{itemize}
    \item \textbf{True Positive (TP):} Predicted box with IoU $\geq \tau$ with a ground truth box
    \item \textbf{False Positive (FP):} Predicted box with IoU $< \tau$ with all ground truth boxes, or duplicate detection of already-matched ground truth
    \item \textbf{False Negative (FN):} Ground truth box not matched by any prediction
\end{itemize}

\subsection{Precision and Recall}

Precision measures the fraction of predictions that are correct:

\begin{equation}
    \text{Precision} = \frac{TP}{TP + FP}
\end{equation}

Recall measures the fraction of ground truths that are detected:

\begin{equation}
    \text{Recall} = \frac{TP}{TP + FN}
\end{equation}

High precision indicates few false alarms; high recall indicates few missed objects. There is typically a trade-off between precision and recall controlled by the confidence threshold.

\subsection{F1 Score}

The F1 score is the harmonic mean of precision and recall, providing a single metric that balances both:

\begin{equation}
    \text{F1} = 2 \cdot \frac{\text{Precision} \cdot \text{Recall}}{\text{Precision} + \text{Recall}} = \frac{2 \cdot TP}{2 \cdot TP + FP + FN}
\end{equation}

F1 ranges from 0 to 1, with higher values indicating better overall performance.

\subsection{Average Precision (AP)}

Average Precision summarizes the precision-recall curve by computing the area under the curve:

\begin{equation}
    \text{AP} = \int_0^1 p(r) \, dr
\end{equation}

where $p(r)$ is precision at recall level $r$. In practice, AP is computed using the 11-point or all-point interpolation method.

\subsection{Mean Average Precision (mAP)}

For multi-class detection, mAP averages AP across all classes:

\begin{equation}
    \text{mAP} = \frac{1}{C} \sum_{c=1}^{C} \text{AP}_c
\end{equation}

Common variants include:
\begin{itemize}
    \item \textbf{mAP@0.5:} AP computed at IoU threshold 0.5
    \item \textbf{mAP@0.5:0.95:} Average of AP at IoU thresholds 0.5, 0.55, ..., 0.95 (COCO metric)
\end{itemize}

\subsection{Inference Speed}

Inference speed is measured in:
\begin{itemize}
    \item \textbf{Milliseconds per image:} Time to process a single image
    \item \textbf{Frames Per Second (FPS):} Number of images processed per second ($\text{FPS} = \frac{1000}{\text{ms per image}}$)
\end{itemize}

Real-time detection typically requires $\geq$30 FPS (33 ms per image).

%% ============================================================================
\section{Summary}
\label{sec:theory_summary}
%% ============================================================================

This chapter has established the theoretical foundations for the thesis:

\begin{enumerate}
    \item \textbf{CNN Fundamentals:} Convolution, pooling, activation functions, batch normalization, and residual connections that enable deep feature extraction.
    
    \item \textbf{YOLO11 Architecture:} Backbone (C3k2, SPPF), neck (PANet, C2PSA), and anchor-free detection head components that together achieve efficient, accurate detection.
    
    \item \textbf{Loss Functions:} CIoU and DFL for box regression, focal loss for classification, combined to guide effective training.
    
    \item \textbf{Evaluation Metrics:} Formal definitions of IoU, precision, recall, F1, and mAP that enable rigorous performance assessment.
\end{enumerate}

The following chapter applies these theoretical foundations to describe our dataset and methodology for Bosphorus maritime vessel detection.

\include{chapters/chapter4_methodology}
\include{chapters/chapter5_experiments}
\include{chapters/chapter6_discussion}
\include{chapters/chapter7_conclusion}

%% ----------------------------------------------------------------------------
%% BACK MATTER
%% ----------------------------------------------------------------------------
\backmatter

% Bibliography
\bibliographystyle{plainnat}
\bibliography{bibliography}
\addcontentsline{toc}{chapter}{Bibliography}

% Appendices
\begin{appendices}
\include{appendices/appendix_a_dataset}
\include{appendices/appendix_b_training_logs}
\include{appendices/appendix_c_code}
\include{appendices/appendix_d_results}
\end{appendices}

\end{document}
